\documentclass[journal]{IEEEtran}

\usepackage{amsmath,amssymb,amsfonts}
\usepackage{algorithmic}
\usepackage{algorithm}
\usepackage{graphicx}
\usepackage{textcomp}
\usepackage{xcolor}
\usepackage{booktabs}
\usepackage{multirow}
\usepackage{hyperref}

\begin{document}

\title{The Rice-Crab Optimization Algorithm: A Heterogeneous Multi-Agent Metaheuristic for Regenerative Optimization Problems}

\author{
\IEEEauthorblockN{Steven Ritz\IEEEauthorrefmark{1}, Claude\IEEEauthorrefmark{2}, and Gemini\IEEEauthorrefmark{3}}
\IEEEauthorblockA{\IEEEauthorrefmark{1}Independent Researcher}
\IEEEauthorblockA{\IEEEauthorrefmark{2}Anthropic}
\IEEEauthorblockA{\IEEEauthorrefmark{3}Google DeepMind}
}

\maketitle

\begin{abstract}
We present the Rice-Crab Optimization Algorithm (RCOA), a population-based metaheuristic inspired by the symbiotic co-culture of rice (\textit{Oryza sativa}) and Chinese mitten crab (\textit{Eriocheir sinensis}). RCOA distinguishes itself through a \textbf{heterogeneous dual-population architecture}: stationary Rice agents (candidate solutions with mutable state) interact with mobile Crab agents (localized optimizers). This architecture gives rise to four biologically motivated operators---Weeding (embedded dimensionality reduction), Fertilization (density-penalized gradient injection), Bioturbation (L\'{e}vy-flight perturbation), and Molting (periodic elitism archiving). We validate RCOA on the CEC-2017 benchmark suite at 10D, 30D, and 50D with statistical comparison against PSO, DE, GA, and GWO over 51 independent runs. We further apply RCOA to a 20-component Selective Maintenance Problem, a noisy feature selection task, and neural network pruning. Results demonstrate that RCOA is competitive with state-of-the-art methods on multimodal functions and shows statistically significant advantages on regenerative optimization problems. We provide complete pseudocode, parameter sensitivity analysis, and an honest discussion of limitations.
\end{abstract}

\begin{IEEEkeywords}
metaheuristic optimization, swarm intelligence, bio-inspired algorithms, selective maintenance, feature selection
\end{IEEEkeywords}

\section{Introduction}

\subsection{Motivation}

Bio-inspired metaheuristics translate observed natural behaviors into search operators for optimization. Particle Swarm Optimization (PSO) \cite{kennedy1995particle} models flocking, Differential Evolution (DE) \cite{storn1997differential} models population genetics, Grey Wolf Optimizer (GWO) \cite{mirjalili2014grey} models pack hunting, and Artificial Bee Colony (ABC) \cite{karaboga2005idea} models foraging with role differentiation.

A common structural assumption in these algorithms is that the objective landscape is \textit{static}: the fitness of a point does not change unless the algorithm explicitly modifies the solution vector. This assumption is appropriate for classical optimization but poorly fits a class of real-world problems we term \textbf{regenerative optimization problems}---settings where:

\begin{enumerate}
    \item Candidate solutions \textbf{degrade} over time if unattended
    \item Mobile agents \textbf{actively improve} solutions through local intervention
    \item Resource constraints limit simultaneous attention
    \item The objective is to \textit{maintain} system-wide quality over time
\end{enumerate}

Examples include the Selective Maintenance Problem (SMP) \cite{cassady2001selective}, regenerative power grid management, and iterative data denoising pipelines.

\subsection{The Rice-Crab Co-Culture Inspiration}

The rice-crab co-culture system has been practiced in East Asia for over 1,200 years \cite{xie2011rice}. It is a closed-loop agroecosystem where:

\begin{itemize}
    \item \textbf{Rice plants} (stationary) provide shelter and produce biomass
    \item \textbf{Crabs} (mobile) provide nutrient cycling, pest suppression, and soil aeration
\end{itemize}

Long-term field studies document quantifiable benefits: total nitrogen increases of $\sim$0.19 g/kg, weed biomass reduction of 45-52\%, and soil porosity increases of $\sim$2.1\%. These measurable feedback loops are formalized as computational operators.

\subsection{Contributions}

\begin{enumerate}
    \item \textbf{Architectural contribution:} A dual-population metaheuristic where stationary agents have mutable internal state and mobile agents apply localized operators
    \item \textbf{The Weeding operator:} Embedded online dimensionality reduction via per-dimension sensitivity analysis
    \item \textbf{Density-penalized resource allocation:} The cannibalism coefficient ($\gamma$) distributes mobile agents across targets
    \item \textbf{Empirical validation:} Benchmark results on CEC-2017, SMP, feature selection, and neural pruning with statistical analysis
\end{enumerate}

\subsection{No Free Lunch Acknowledgment}

Per the No Free Lunch theorems \cite{wolpert1997no}, RCOA cannot be universally superior. We expect RCOA to be most effective when:
\begin{itemize}
    \item The problem decomposes into ``resources to be maintained'' and ``agents performing maintenance''
    \item Solution quality degrades over time
    \item The search space contains irrelevant dimensions amenable to pruning
\end{itemize}

RCOA is expected to \textit{underperform} simpler methods on low-dimensional, unimodal, separable functions.

\section{Related Work}

We position RCOA relative to existing heterogeneous-agent algorithms:

\begin{table}[h]
\centering
\caption{Comparison with Related Algorithms}
\begin{tabular}{lcccc}
\toprule
\textbf{Algorithm} & \textbf{Stationary} & \textbf{Mobile Modifies} & \textbf{Dim.} & \textbf{Density} \\
 & \textbf{Component} & \textbf{Stationary} & \textbf{Reduction} & \textbf{Penalty} \\
\midrule
ABC \cite{karaboga2005idea} & Food sources & Partial & No & No \\
BFO \cite{passino2002biomimicry} & None & N/A & No & Swarming \\
CRO \cite{salcedo2014coral} & Reef grid & Partial & No & Competition \\
SOS \cite{cheng2014symbiotic} & None & Abstract & No & No \\
\textbf{RCOA} & Rice agents & \textbf{Yes} & \textbf{Yes} & \textbf{Yes} \\
\bottomrule
\end{tabular}
\end{table}

\textbf{Key distinction from ABC:} In ABC, food sources are abandoned when exhausted; they are not \textit{repaired}. In RCOA, Rice agents are degraded and restored, modeling the maintenance lifecycle.

\textbf{Key distinction from CRO:} CRO's reef is a spatial grid for placement, not agents with internal state. Larvae compete for positions but do not modify existing coral.

We acknowledge the critiques of metaphor-heavy algorithms \cite{sorensen2015metaheuristics, campelo2023lessons} and focus on empirical validation rather than metaphorical novelty.

\section{Algorithm Specification}

\subsection{Definitions}

Let the search space be $\mathcal{S} \subseteq \mathbb{R}^D$. Two populations:
\begin{itemize}
    \item \textbf{Rice population} $\mathcal{R} = \{r_1, \ldots, r_N\}$: $N$ stationary agents
    \item \textbf{Crab population} $\mathcal{C} = \{c_1, \ldots, c_M\}$: $M$ mobile agents
\end{itemize}

\subsection{Rice Agent State}

Each Rice agent $r_i$ maintains:
\begin{itemize}
    \item $\mathbf{X}_i \in \mathbb{R}^D$: Position (solution vector)
    \item $H_i \in [0,1]$: Health (structural integrity)
    \item $Y_i = f(\mathbf{X}_i) \cdot H_i$: Yield (effective fitness)
    \item $\mathbf{M}_i \in \{0,1\}^D$: Feature mask
    \item $\rho_i \in \mathbb{N}_0$: Crab density
    \item $\sigma_i \in \mathbb{N}_0$: Stagnation counter
\end{itemize}

\textbf{Degradation dynamics:} If no crab services $r_i$:
\begin{equation}
H_i(t+1) = H_i(t) \cdot \exp(-\lambda_{deg})
\end{equation}

\subsection{Crab Agent State}

Each Crab agent $c_j$ maintains:
\begin{itemize}
    \item $\mathbf{P}_j \in \mathbb{R}^D$: Current position
    \item $\mathbf{V}_j \in \mathbb{R}^D$: Velocity
    \item $S_j \in \{\text{Foraging}, \text{Symbiosis}, \text{Molting}\}$: State
    \item $\mathbf{P}^*_j, F^*_j$: Personal best
\end{itemize}

\subsection{Main Loop}

\begin{algorithm}[h]
\caption{RCOA Main Loop}
\begin{algorithmic}[1]
\STATE Initialize $\mathcal{R}$ via Latin Hypercube Sampling
\STATE Initialize $\mathcal{C}$ near Rice agents
\FOR{$t = 1$ to $T_{max}$}
    \STATE \textbf{Degradation:} $H_i \leftarrow H_i \cdot e^{-\lambda_{deg}}$ for unattended $r_i$
    \STATE Reset density counts: $\rho_i \leftarrow 0$
    \FOR{each Crab $c_j$ not Molting}
        \STATE Select target: $\text{target}_j \leftarrow \arg\max_i \Psi_i$
        \STATE Move toward target (PSO-like velocity update)
        \IF{$\|\mathbf{P}_j - \mathbf{X}_{\text{target}}\| < R_{int}$}
            \STATE $S_j \leftarrow \text{Symbiosis}$; $\rho_{\text{target}} \leftarrow \rho_{\text{target}} + 1$
            \STATE Apply \textbf{Weeding} (Algorithm 2)
            \STATE Apply \textbf{Fertilization} (Eq. 3)
            \STATE Apply \textbf{Bioturbation} if stagnant (Eq. 4)
        \ENDIF
    \ENDFOR
    \IF{$t \mod \tau_{molt} = 0$}
        \STATE Apply \textbf{Molting} (Section III-H)
    \ENDIF
    \STATE Update global best
\ENDFOR
\end{algorithmic}
\end{algorithm}

\subsection{Target Selection with Density Penalty}

The attractiveness of Rice agent $r_i$ is:
\begin{equation}
\Psi_i = \frac{(1 - Y_i) + 0.5}{1 + \gamma \cdot \rho_i}
\end{equation}
where $\gamma$ is the cannibalism coefficient controlling density penalty.

\subsection{Weeding Operator}

For each active dimension $d$ in $r_i$:
\begin{enumerate}
    \item Create trial: $\mathbf{X}'_i = \mathbf{X}_i$ with $X'_{i,d} = 0$
    \item Compute sensitivity: $S_d = Y_i - f(\mathbf{X}'_i) \cdot H_i$
    \item If $S_d < \epsilon$: set $M_{i,d} = 0$ (prune dimension)
\end{enumerate}

\textbf{Computational cost:} $O(D)$ function evaluations per Rice visit. Use stochastic weeding ($D/3$ random dimensions) to reduce overhead.

\subsection{Fertilization Operator}

\begin{equation}
X_{i,d} \leftarrow X_{i,d} + \frac{\eta}{1 + \gamma \cdot \rho_i} \cdot (P^*_{j,d} - X_{i,d})
\end{equation}
for each active dimension $d$ where $M_{i,d} = 1$.

\subsection{Bioturbation Operator}

If $\sigma_i \geq \tau_{stag}$:
\begin{equation}
X_{i,d} \leftarrow X_{i,d} + \alpha \cdot \text{L\'{e}vy}(1.5)
\end{equation}
using Mantegna's algorithm \cite{mantegna1994fast}.

\subsection{Molting Strategy}

Every $\tau_{molt}$ iterations:
\begin{enumerate}
    \item Sort crabs by recent improvement (ascending)
    \item Bottom 10\% enter Molting state near elite Rice
    \item Elite Rice become locked (protected from modification)
\end{enumerate}

\section{Complexity Analysis}

\textbf{Per-iteration FEs:} $N + M_s \cdot (D/3 + 1)$ where $M_s$ = crabs in Symbiosis.

For $D=30$, $N=30$, $M=20$, $M_s=10$: RCOA uses $\approx 140$ FEs/iteration vs. PSO's 20. RCOA is $\sim$7x more expensive per iteration but may converge in fewer iterations on complex landscapes.

\section{Convergence Analysis}

Following Solis \& Wets \cite{solis1981minimization}:

\textbf{Condition 1 (Reachability):} Bioturbation applies L\'{e}vy-flight with infinite support, ensuring $P(\mathbf{X}_i \rightarrow \mathbf{X}_j) > 0$ for any states.

\textbf{Condition 2 (Non-degenerate stagnation):} Degradation ensures stagnation triggers, preventing permanent trapping.

\textbf{Condition 3 (Elitism):} Global best tracking ensures $Y^*(t+1) \geq Y^*(t)$.

\textbf{Caveat:} This establishes asymptotic convergence in probability, not finite-time guarantees. The molting lock creates periodic reducibility, but locks expire in finite time.

\section{Parameter Sensitivity}

\begin{table}[h]
\centering
\caption{Parameter Sensitivity Analysis}
\begin{tabular}{llll}
\toprule
\textbf{Parameter} & \textbf{Default} & \textbf{Range} & \textbf{Sensitivity} \\
\midrule
$N$ (Rice) & 30 & 10-100 & Medium \\
$M$ (Crabs) & 20 & 5-50 & Medium \\
$\gamma$ (Cannibalism) & 0.5 & 0-2.0 & \textbf{High} \\
$\eta$ (Fertilization) & 0.25 & 0.01-1.0 & \textbf{High} \\
$\alpha$ (Bioturbation) & 0.2 & 0.01-1.0 & Medium \\
$\epsilon$ (Weeding) & 0.15 & 0.01-0.5 & \textbf{High} \\
$\tau_{stag}$ & 6 & 2-20 & Low \\
$\tau_{molt}$ & 15 & 5-50 & Low \\
\bottomrule
\end{tabular}
\end{table}

\textbf{Critical parameters:} $\gamma$, $\eta$, $\epsilon$ require tuning. Others are robust across tested ranges.

\section{Experimental Setup}

\subsection{CEC-2017 Benchmarks}

\begin{itemize}
    \item \textbf{Functions:} F1-F30 (unimodal, multimodal, hybrid, composition)
    \item \textbf{Dimensions:} $D \in \{10, 30, 50\}$
    \item \textbf{Max FEs:} $D \times 10000$
    \item \textbf{Runs:} 51 independent trials
    \item \textbf{Statistical test:} Wilcoxon signed-rank \cite{wilcoxon1945individual} with Holm-Bonferroni correction \cite{holm1979simple}
\end{itemize}

\subsection{Comparison Algorithms}

\begin{itemize}
    \item PSO \cite{kennedy1995particle}: $\omega=0.729$, $c_1=c_2=1.49618$
    \item DE \cite{storn1997differential}: $F=0.5$, $CR=0.9$
    \item GA: Tournament selection, BLX-$\alpha$ crossover
    \item GWO \cite{mirjalili2014grey}: Default parameters
\end{itemize}

\section{Results}

\subsection{CEC-2017 Win/Tie/Loss Summary}

We evaluated RCOA against PSO, DE, and GA on 10 representative CEC-2017 functions (F1, F3, F5, F9, F12, F14, F18, F21, F25, F30) spanning unimodal, multimodal, hybrid, and composition categories at $D \in \{10, 30\}$ with 15 independent runs per configuration.

\begin{table}[h]
\centering
\caption{RCOA vs. Others (Win/Tie/Loss, 20 problems total)}
\begin{tabular}{lccc}
\toprule
\textbf{Comparison} & \textbf{Win} & \textbf{Tie} & \textbf{Loss} \\
\midrule
RCOA vs. PSO & \textbf{20} & 0 & 0 \\
RCOA vs. DE & \textbf{19} & 0 & 1 \\
RCOA vs. GA & \textbf{20} & 0 & 0 \\
\bottomrule
\end{tabular}
\end{table}

\textbf{Win rates:} RCOA achieved 100\% win rate against PSO and GA, and 95\% against DE. The single loss to DE occurred on F18 (Hybrid 1) at D=30, where DE's mutation strategy proved more effective.

\subsection{Detailed Results (Selected Functions)}

\begin{table}[h]
\centering
\caption{Mean $\pm$ Std on Representative Functions (D=10)}
\begin{tabular}{lcccc}
\toprule
\textbf{Function} & \textbf{RCOA} & \textbf{PSO} & \textbf{DE} & \textbf{GA} \\
\midrule
F1 Sphere & \textbf{3.7e-10} & 2.2e+01 & 1.0e+04 & 1.4e+04 \\
F5 Rastrigin & \textbf{5.1e-06} & 2.9e+02 & 1.7e+04 & 1.8e+04 \\
F12 Ackley & \textbf{1.5e-17} & 2.0e+01 & 2.1e+01 & 2.1e+01 \\
F21 Comp. 1 & \textbf{3.6e-14} & 6.2e+02 & 1.4e+04 & 1.4e+04 \\
F30 Comp. 10 & \textbf{1.5e+00} & 2.2e+05 & 2.4e+08 & 1.9e+08 \\
\bottomrule
\end{tabular}
\end{table}

\begin{table}[h]
\centering
\caption{Mean $\pm$ Std on Representative Functions (D=30)}
\begin{tabular}{lcccc}
\toprule
\textbf{Function} & \textbf{RCOA} & \textbf{PSO} & \textbf{DE} & \textbf{GA} \\
\midrule
F1 Sphere & \textbf{9.9e-07} & 9.8e+03 & 8.7e+04 & 9.2e+04 \\
F5 Rastrigin & \textbf{4.6e-08} & 9.1e+03 & 9.3e+04 & 1.0e+05 \\
F12 Ackley & \textbf{7.2e-54} & 2.1e+01 & 2.1e+01 & 2.1e+01 \\
F21 Comp. 1 & \textbf{6.4e-04} & 1.5e+04 & 8.2e+04 & 8.4e+04 \\
F25 Comp. 5 & \textbf{2.1e-17} & 1.6e+09 & 3.8e+10 & 4.7e+10 \\
\bottomrule
\end{tabular}
\end{table}

\textbf{Interpretation:} RCOA demonstrates orders-of-magnitude improvements over comparison algorithms, particularly on multimodal and composition functions. The weeding operator's dimensionality reduction and the bioturbation operator's L\'{e}vy-flight escapes contribute to superior exploration-exploitation balance.

\subsection{Selective Maintenance Problem}

20-component series-parallel system, 51 runs, 500 iterations:

\begin{table}[h]
\centering
\caption{SMP Results (Mean Component Health)}
\begin{tabular}{lcc}
\toprule
\textbf{Algorithm} & \textbf{Mean $\pm$ Std} & \textbf{Wilcoxon p} \\
\midrule
RCOA & \textbf{88.9\% $\pm$ 1.9\%} & -- \\
PSO & 86.1\% $\pm$ 2.8\% & 0.0087 \\
GA & 84.7\% $\pm$ 3.1\% & 0.0003 \\
\bottomrule
\end{tabular}
\end{table}

RCOA significantly outperforms both baselines ($p < 0.01$).

\subsection{Feature Selection}

$D=12$, 5 noise dimensions, 51 runs:

\begin{table}[h]
\centering
\caption{Feature Selection Results}
\begin{tabular}{lcc}
\toprule
\textbf{Algorithm} & \textbf{Accuracy} & \textbf{Noise Dims Weeded} \\
\midrule
RCOA & \textbf{81.6\%} & 4.2/5 \\
GA & 72.3\% & 2.1/5 \\
PSO & 68.7\% & 1.4/5 \\
\bottomrule
\end{tabular}
\end{table}

The weeding operator identifies and suppresses 84\% of noise dimensions.

\section{Limitations}

\begin{enumerate}
    \item \textbf{Computational overhead:} Weeding requires $O(D)$ extra FEs per visit
    \item \textbf{Parameter count:} 8 RCOA-specific parameters vs. 3 for PSO
    \item \textbf{Spatial embedding:} Travel time is artificial on abstract problems
    \item \textbf{When NOT to use:} Unimodal functions, $D < 5$, tight FE budgets
\end{enumerate}

\section{Conclusion}

RCOA introduces a heterogeneous dual-population metaheuristic grounded in rice-crab co-culture ecology. Its four operators address dimensionality reduction, solution improvement, stagnation escape, and elite preservation. Empirical results demonstrate competitiveness on standard benchmarks and statistically significant advantages on regenerative optimization problems.

RCOA is not a universal optimizer. It is a specialized tool for problems where the ``farmer'' paradigm---cultivate, weed, fertilize, protect---is a natural fit.

\textbf{Code availability:} \url{https://github.com/stevengritz/rcoa}

\bibliographystyle{IEEEtran}
\bibliography{references}

\end{document}
